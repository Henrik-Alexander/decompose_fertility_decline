\documentclass[a4paper, 12pth]{article}
\usepackage{amsmath}

\title{Decomposition of the TFR}
\author{Henrik-Alexander Schubert}
\newcommand\pd[2]{\frac{\partial #1}{\partial #2}}

\begin{document}

\maketitle
%=============================================================

The TFR is defined as the sum of age specific fertility rates ($f(x)$), which is the ratio of number of births to the person-years exposed to the event, 
\begin{equation}
    f(x) = \frac{births(x)}{exposure(x)}.
\end{equation}
Therefore the age-specific fertility rate can be rewritten into the population subgroups ($p$) that contribute to it, so that the population age-specific fertility rate is the sum over all subgroups $p$,
\begin{equation}
  f(x) = \sum_{x=15}^{55} \left[ \frac{\sum_i^p births_i(x)}{\sum_i^p exposure_i(x)} \right].
\end{equation}
Since the estimation is additive, we can rewrite the equation as weighted average: 
\begin{equation}
    TFR = \sum_{x = 1}^{55} \sum_{i = 1}^{p} \underbrace{f_i(x)}_{Behavioral} \times \underbrace{weight_i}_{Composition}
\end{equation}
The equation above can be used to decompose the difference in the TFR into the behavioral component stemming from a difference in $f_i(x)$ and a compositional component $weight_i$.

\begin{equation}
    \Delta TFR = \sum_{x = 1}^{55}
    \underbrace{\pd{TFR}{f(x)} \cdot \frac{weight(x)_i + weight(x)_j}{2}}_{\Delta Rate} +
    \underbrace{\pd{TFR}{weight} \cdot \frac{f(x)_i + f(x)_j}{2}}_{\Delta Composition},
\end{equation}
which is the observed rate difference between the two groups multiplied with the initial response of the TFR to a change in the age-specific fertility rate, and the initial change in the TFR to a change in the weight of a group.

\end{document}
